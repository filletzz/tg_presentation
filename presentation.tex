\documentclass[hyperref={pdfpagelabels=false}]{beamer}
\usepackage[english]{babel}
\usepackage[utf8]{inputenc}
\usepackage{amsmath, amsthm, amsfonts, amssymb}
\theoremstyle{definition}
\newtheorem{dfn}{Definition}
\newtheorem*{exc}{Example}
\newtheorem{lem}{Lemma}
\newtheorem*{rem}{Remark}
\newtheorem{thm}{Theorem}
\newtheorem{cor}{Corollar}
\newtheorem{ntn}{Notation}
\newtheorem{rep}{Repetition}

\mode<presentation> { \usetheme{Montpellier} }

\title{On the metric dimension, the upper dimension\\ and the resolving number of graphs}

\author{Fátima Martínez Macías, Petr Chmela\v{r}, Felix Reihl}
\date{}

\begin{document}
\begin{frame}
	\titlepage
\end{frame} 
% ----
\begin{frame}
	\tableofcontents
\end{frame} 
% ----
\section{Introduction}
% Petr
\begin{frame}
\end{frame}
% ----
\section{Randomly k-dimensional graphs}
% Fátima
\begin{frame}
\end{frame}
% ----
\section{Realization}
% ----
\subsection{The metric dimension and the upper dimension}
% Felix
\begin{frame}
	\frametitle{Definitions}
	\begin{dfn}[Grid graph]
		Let $G_l$ be the quadratic grid graph with $l$ vertices in both directions. The \emph{coordinates} $(x_1,x_2)$ of a vertex $x$ start with $(0,0)$ at the bottom left corner.
	\end{dfn}
	\begin{dfn}[Quadrants and Diagonals]
		The \emph{quadrants} of a vertex $x$ in a grid graph are $Q_1$ (the top right sector of $x$), $Q_2$ (top left), $Q_3$ (bottom left) and $Q_4$ (bottom right).

		The \emph{diagonals} of a grid graph are $D_i = \{ (x_1,x_2) \mid x_1 + x_2 = i \}$. A pair $\{x,y\}$ is a \emph{diagonal pair} if a diagonal exists that contains both.
	\end{dfn}
\end{frame}
% ---
\begin{frame}
	$R(x,y)$ is the set of vertices that resolve $x$ and $y$.
	\begin{lem}
		Let $\{x,y\}$ be a diagonal pair with $d(x,y) = 2$. Then
		\[ R(x,y) = Q_2(x) \cup Q_4(y) \]
	\end{lem}
	\begin{dfn}[S-uniqueness]
		For a resolving set $S$, a pair $\{x,y\}$ is \emph{S-unique} if there is a unique vertex $x \in S$ resolving $x$ and $y$. $u$ and $\{x,y\}$ are \emph{associated to each other}.
	\end{dfn}
\end{frame}
% ---
\begin{frame}
	\begin{lem}
		Let $S$ be a resolving set and let $\{x,y\} = \{(x_1,x_2),(y_1,y_2)\}$ be a S-unique diagonal pair with associated vertex $u$ such that $d(x,y)>2$.

		Then, there exist excactly $y_1 - x_1$ S-unique diagonal pairs with associated vertex $u$ and distance $2$.
	\end{lem}
\end{frame}
% ----
\subsection{The resolving number}
%Petr
\begin{frame}
\end{frame}
% ----
\end{document}
