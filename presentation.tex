\documentclass[hyperref={pdfpagelabels=false}]{beamer}
\usepackage[english]{babel}
\usepackage[utf8]{inputenc}
\usepackage{amsmath, amsthm, amsfonts, amssymb}
\usepackage{tikz}
\usepackage{enumerate}
\usepackage{mathtools}
\theoremstyle{definition}
\newtheorem{dfn}{Definition}
\newtheorem*{exc}{Example}
\newtheorem{ntn}{Notation}
\newtheorem{rep}{Repetition}
\newtheorem{obs}{Observation}
\newtheorem*{rem}{Remark}
\theoremstyle{theorem}
\newtheorem{lem}{Lemma}
\newtheorem{thm}{Theorem}
\newtheorem{cor}{Corollar}
\newtheorem{clm}{Claim}

\mode<presentation> { \usetheme{Montpellier} }

\title{On the metric dimension, the upper dimension\\ and the resolving number of graphs}

\author{Fátima Martínez Macías, Petr Chmela\v{r}, Felix Reihl}
\date{}

\begin{document}
\begin{frame}
	\titlepage
\end{frame} 
% ----
\begin{frame}
	\tableofcontents
\end{frame} 
% ----
\section{Introduction}
% Petr
%\begin{frame}
%\end{frame}
% ----
\section{Randomly k-dimensional graphs}
% Fátima
%\begin{frame}
%\end{frame}
% ----
\section{Realization}
% ----
\subsection{The metric dimension and the upper dimension}
% Felix
\begin{frame}
	\frametitle{Definitions}
	\begin{dfn}[Grid graph]
		Let $G_\ell$ be the quadratic grid graph with $\ell$ vertices in both directions. The \emph{coordinates} $(x_1,x_2)$ of a vertex $x$ start with $(0,0)$ at the bottom left corner.
	\end{dfn}
	\begin{dfn}[Quadrants and Diagonals]
		The \emph{quadrants} of a vertex $x$ in a grid graph are $Q_1$ (the top right sector of $x$), $Q_2$ (top left), $Q_3$ (bottom left) and $Q_4$ (bottom right).

		The \emph{diagonals} of a grid graph are $D_i = \{ (x_1,x_2) \mid x_1 + x_2 = i \}$. A pair $\{x,y\}$ is a \emph{diagonal pair} if a diagonal exists that contains both.
	\end{dfn}
\end{frame}
% ---
\begin{frame}
	$R(x,y)$ is the set of vertices that resolve $x$ and $y$.
	\begin{lem}
		Let $\{x,y\}$ be a diagonal pair with $d(x,y) = 2$. Then
		\[ R(x,y) = Q_2(x) \cup Q_4(y) \]
	\end{lem}
\end{frame}
% ---
\begin{frame}
	\begin{dfn}[S-uniqueness]
		For a resolving set $S$, a pair $\{x,y\}$ is \emph{S-unique} if there is a unique vertex $x \in S$ resolving $x$ and $y$. $u$ and $\{x,y\}$ are \emph{associated to each other}.
	\end{dfn}
	\begin{obs}
		If $\{x,y\}$ is S-unique with associated vertex $u$ and $\{r,s\}$ has $R(r,s) \subseteq R(x,y)$, then $\{r,s\}$ is also S-unique with associated vertex $u$.
	\end{obs}
\end{frame}
% ---
\begin{frame}
	\begin{lem}
		Let $S$ be a resolving set and let $\{x,y\} = \{(x_1,x_2),(y_1,y_2)\}$ be a S-unique diagonal pair with associated vertex $u$ such that $d(x,y)>2$.

		Then, there exist excactly $y_1 - x_1$ S-unique diagonal pairs with associated vertex $u$ and distance $2$.
	\end{lem}
%	\begin{center}
%		\begin{tikzpicture}[scale=0.4]
%			\tikzset{dot/.style={circle,fill=black,inner sep=0,minimum size=0.2em}}
%			% horizontal
%			\draw (0,0) to (8,0);
%			\draw (0,1) to (8,1);
%			\draw (0,2) to (8,2);
%			\draw (0,3) to (8,3);
%			\draw (0,4) to (8,4);
%			\draw (0,5) to (8,5);
%			\draw (0,6) to (8,6);
%			\draw (0,7) to (8,7);
%			\draw (0,8) to (8,8);
%			% vertical
%			\draw (0,0) to (0,8);
%			\draw (1,0) to (1,8);
%			\draw (2,0) to (2,8);
%			\draw (3,0) to (3,8);
%			\draw (4,0) to (4,8);
%			\draw (5,0) to (5,8);
%			\draw (6,0) to (6,8);
%			\draw (7,0) to (7,8);
%			\draw (8,0) to (8,8);
%			% nodes
%			\node [dot, label=above right:\tiny y] (y) at (6,3) {};
%		\end{tikzpicture}
%	\end{center}
\end{frame}
% ---
\begin{frame}
	\begin{lem}
		Let $S$ be a resolving set and let $\{x,y\}$ be a S-unique diagonal pair with associated vertex $u$ such that $d(x,y)=2$. If there exist two S-unique diagonal pairs in the same row (column) as $\{x,y\}$ with associated vertices $v$ and $w$, and $u \neq v$, then $v = w$.
	\end{lem}
\end{frame}
% ---
\begin{frame}
	\begin{thm}
		For every pair $a,b$ of integers with $2 \leq a \leq b$, there exists a connected graph $G$ with $\dim(G) = a$ and $\dim^+(G) = b$.
	\end{thm}
\end{frame}
% ---
\begin{frame}
	\frametitle{Claim A}
	\begin{ntn}
		Let $H_\ell$ be the grid graph $G_\ell$, with a triangle attached at $(0,0)$. The vertices of the triangle are called $\alpha, \beta$.
	\end{ntn}
	\begin{clm}
		$\dim(H_\ell) = 2$ and $\dim^+(H_\ell) = 2\ell - 2$.
	\end{clm}
	\begin{block}{Steps of the proof}
		\begin{enumerate}
			\item $\dim(H_\ell) = 2$
			\item $\dim^+(H_\ell) \geq 2\ell - 2$
			\item $\dim^+(H_\ell) \leq 2\ell - 2$
		\end{enumerate}
	\end{block}
\end{frame}
% ---
\begin{frame}
	\frametitle{Step 3}
	\begin{itemize}
		\item $S$ is minimal, therefore every $u \in S$ has an associated S-unique pair $p(u)$.
		%\item $\{\beta, (0,0)\}$ is not a S-unique pair (every vertex of $G_\ell$ resolves it).
		\item $\alpha$ resolves every non-diagonal pair.
		\item Therefore, every $u \in S \setminus \{\alpha\}$ has an associated S-unique diagonal pair $p(u)$.
		\item Lemma 3: These pairs are at distance 2 to each other.
		\item Lemma 4: For all pairs in the same row (or column), at most \emph{two} distinct vertices are associated to them.
		\item Lemma 1: In the first row (or column) there is at most \emph{one} element of the resolving set. % S minimal
	\end{itemize}
\end{frame}
% ---
\begin{frame}
	\frametitle{Step 3, cont.}
	\begin{itemize}
		\item Since there are $\ell - 1$ rows (and columns), there are at most this many S-unique pairs that can be associated to the vertices of $S \setminus \{\alpha\}$:
			\[\underbrace{2}_{\substack{\text{at most vertices}\\\text{for all pairs}\\\text{in a row}}}
			(\underbrace{\ell - 2}_{\substack{\text{rows without}\\\text{first one}}})
		+ \underbrace{1}_{\substack{\text{at most 1}\\\text{vertex in the first}\\\text{row and column}}}\]
		\item Therefore $|S \setminus \{\alpha\}| = |S| - 1 \leq 2(\ell - 2) + 1$.
	\end{itemize}
\end{frame}
% ---
\begin{frame}
	\frametitle{Claim B}
	\begin{ntn}
		Let $H_{\ell,m}$ be the grid graph $G_\ell$ ($\ell \geq 3$), with a set of $m \geq 2$ pendant vertices $\alpha_1, \dotsc, \alpha_m$ attached at $(0,0)$.
	\end{ntn}
	\begin{clm}
		$\dim(H_{\ell,m}) = m + 1$ and $\dim^+(H_{\ell,m}) = m + 2\ell - 4$.
	\end{clm}
\end{frame}
% ----
\subsection{The resolving number}
%Petr
%\begin{frame}
%\end{frame}
% ----
\end{document}
