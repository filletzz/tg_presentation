\documentclass[hyperref={pdfpagelabels=false}]{beamer}
\usepackage[ngerman]{babel}
\usepackage{lmodern}
\usepackage{graphicx}
\usepackage{caption}
\usepackage{diagbox}
\usepackage[utf8]{inputenc}
\usepackage{tikz}
\mode<presentation> { \usetheme{Montpellier} }
\newcommand{\meet}{\wedge}
\newcommand{\compose}{\circ}

\title{Solving Size Constraints \\Using Graph Representation}

\author{Felix Reihl}
\date{09.08.2013}

%\beamerdefaultoverlayspecification{<+->}
\begin{document}

\begin{frame}
	\titlepage
\end{frame} 

\begin{frame}
	\tableofcontents
\end{frame} 

\section{Einführung}
%\begin{frame}
%	\frametitle{Motivation}
%\end{frame}
% ---
\subsection{Definition des Problems}
\begin{frame}
	\begin{block}{Grammatik}
		\small
		\[
			\begin{array}{lrl@{\qquad}l}
				i,j,k &&& \mbox{rigide Variablen} \\
				n     & ::=  & 0 \mid 1 \mid \dots & \mbox{feste natürliche Zahl}
				\\
				v,w   & ::=  & n      & \mbox{Größenwert (size value)} \\
									& \mid & i + n  & \mbox{rigide Variable + Offset}  \\
								 & \mid & \infty & \mbox{unendliche Größe}  \\
								& \mid & \max(\vec v) & \mbox{Maximum von $v_1$, \dots, $v_n$} \\ 
				H     & ::= & i < v \mid i \leq v & \mbox{Hypothese} \\
				\Gamma& ::= & \vec H & \mbox{Kontext (Liste von Hypothesen)} \\
			\end{array}
		\]
	\end{block}
\end{frame}
% ---
\begin{frame}
	\begin{block}{Grammatik (Forts.)}
		\small
		\[
			\begin{array}{lrl@{\qquad}l}
				X,Y,Z &&& \mbox{flexible Variablen (zu lösen)} \\
				a,b,c & ::=  & v             &  \mbox{Größenausdruck (size expression)} \\
									& \mid & \max(\vec a) & \mbox{Maximum von $a_1$, \dots, $a_n$} \\ 
								 & \mid & X(\vec a) + n & \mbox{ungelöste Funktion (angewendet} \\
								 &&& \mbox{auf Argumente) + Offset} \\
				C     & ::= & a < b \mid a \leq b & \mbox{Constraint Ungleichung} \\
			\end{array}
		\]
	\end{block}
\end{frame}
% ---
\begin{frame}[fragile]
	\frametitle{Beispiel}
	\tiny
	\begin{verbatim}
  data List +(i : Size) ++(A : Set)
  { nil
  ; cons [j < i] (x : A) (xs : List j A) : List i A
  }

  data Bool { true ; false }
  
  fun T   : Set {}
  fun leq : T -> T -> Bool
  
  fun insert : [i : Size] |i| -> T -> List i T -> List (i + 1) T
                        {- termination order -}
  { insert i x nil            = cons _ x nil
                                  {- X1 -}
  ; insert i x (cons i' y ys) = case leq x y
    { true  -> cons _ x (cons _ y ys)
                 {- X2 -}  {- X3 -}
    ; false -> cons _ y (insert _ x ys)
                 {- X4 -}    {- X5 -}
    }
  }
	\end{verbatim}
	\begin{minipage}{30em}
		\begin{block}{Constraints}
			\tiny
			\begin{minipage}{14em}
				\begin{itemize}
					\item $X_1 < i + 1$
					\item $X_2 < i + 1$ und $i' \leq X_3$
					\item $X_3 < X_2$
				\end{itemize}
			\end{minipage}
			\begin{minipage}{14em}
				\begin{itemize}
					\item $X_4 < i + 1$
					\item $i' \leq X5$
					\item $X_5 < i$
				\end{itemize}
			\end{minipage}
			%		\begin{tabular}{ll}
			%			$X_1 < i + 1$ & $X_2 < i + 1$ und $i' \leq X_3$ \\
			%			$X_3 < X_2$ & $X_4 < i + 1$ \\
			%			$i' \leq X5$ & $X_5 < i$
			%		\end{tabular}
		\end{block}
	\end{minipage}
	\begin{minipage}{10em}
		\begin{block}{Hypothesen}
			\tiny
			\begin{itemize}
				\item $i' < i$
			\end{itemize}
		\end{block}
	\end{minipage}
\end{frame}
% ---
\begin{frame}
	\frametitle{Resultierender Graph}
	\begin{center}
		\begin{tikzpicture}[->,scale=2,inner sep=0.2em]
			\node (x1)  at (0,1) {$X_1$};
			\node (x2) at (0,0) {$X_2$};
			\node (x3) at (1,0) {$X_3$};
			\node (x4) at (1,1) {$X_4$};
			\node (x5) at (2,0) {$X_5$};
			\node (i) at (-1,1) {$i$};
			\node (i') at (-1,0) {$i'$};
			\path
			(x1) edge node {} (i)
			(x2) edge node {} (i)
			(x4) edge[bend right=30] node {} (i)
			(x5) edge node[above] {$-1$} (i)
			(x3) edge node[above] {$-1$} (x2)
			(x5) edge node[above right] {$-1$} (x4)
			(i') edge node[right] {$-1$} (i)
			(i') edge[bend right] node {} (x3)
			(i') edge[bend right=40] node {} (x5)
			;
		\end{tikzpicture}
	\end{center}
\end{frame}
% ---
\section{Vereinfachter Ansatz}
% ---
\begin{frame}
	\begin{block}{Vorerst}
		\begin{itemize}
			\item keine Maxima
			\item alle flexiblen Variablen hängen von den gleichen Signaturen ab: $X\ i\ j\ k,\ Y\ i\ j\ k$
		\end{itemize}
	\end{block}
	\begin{block}{Vorgehen}
		\begin{enumerate}
			\item Vereinfachung der Constraints (\ref{simp})
			\item Modellierung als Graph (\ref{mod})
			\item Transitive Hülle durch Komposition (\ref{trans})
			\item Finden von Widersprüchen innerhalb der Constraints und mit den Hypothesen (\ref{cont})
			\item Algebraische Struktur (\ref{alg})
			\item Finden der Lösung im Graphen (\ref{sol})
			\item Erweitern des Problems (\ref{ext})
		\end{enumerate}
	\end{block}
\end{frame}
% ---
\subsection{Vereinfachung der Constraints}
\label{simp}
\begin{frame}
	\begin{block}{Vereinfachte Formen}
		\begin{align}
			\label{case00} X &< \infty \\
			\label{case03} n &\leq X \\
			\label{case04} X &\leq n \\
			\label{case01} X + n &\leq i \\
			\label{case02} X &\leq i + n \\
			\label{case05} i + n &\leq X \\
			\label{case06} i &\leq X + n \\
			\label{case07} Y + n &\leq X \\
			\label{case08} Y &\leq X + n \\
			\label{case09} Y + n &< X \\
			\label{case10} Y &< X + n
		\end{align}
	\end{block}
\end{frame}
% ---
\subsection{Modellierung als Graph}
\label{mod}
\begin{frame}
	\frametitle{Formen von Kanten}
	Jede Ungleichung wird zu einer Kante im Graphen
	\begin{center}
		\begin{tabular}{lll}
			Fall & Constraint & Kante \\
			\hline
			\ref{case00} & $X < \infty$   &
			\begin{tikzpicture}[->,baseline=-0.25em]
				\node (a) at (0,0) {$X$};
				\node (b) at (1,0) {$\infty$};
				\node (label1) at (0.5,-0.15) {\tiny$0$};
				\node (label2) at (0.5,0.15) {\tiny$<$};
				\draw (a) to (b);
			\end{tikzpicture}
			\\
			\ref{case03} & $n \leq X$     &
			\begin{tikzpicture}[->,baseline=-0.25em]
				\node (a) at (0,0) {$0$};
				\node (b) at (1,0) {$X$};
				\node (label1) at (0.5,-0.15) {\tiny$-n$};
				\node (label2) at (0.5,0.15) {\tiny$\leq$};
				\draw (a) to (b);
			\end{tikzpicture}
			\\
			\ref{case04} & $X \leq n$     &
			\begin{tikzpicture}[->,baseline=-0.25em]
				\node (a) at (0,0) {$0$};
				\node (b) at (1,0) {$X$};
				\node (label1) at (0.5,-0.15) {\tiny$+n$};
				\node (label2) at (0.5,0.15) {\tiny$\leq$};
				\draw (a) to (b);
			\end{tikzpicture}
			\\
			\ref{case01} & $X + n \leq i$ &
			\begin{tikzpicture}[->,baseline=-0.25em]
				\node (a) at (0,0) {$X$};
				\node (b) at (1,0) {$i$};
				\node (label1) at (0.5,-0.15) {\tiny$-n$};
				\node (label2) at (0.5,0.15) {\tiny$\leq$};
				\draw (a) to (b);
			\end{tikzpicture}
			\\
			\ref{case02} & $X \leq i + n$ &
			\begin{tikzpicture}[->,baseline=-0.25em]
				\node (a) at (0,0) {$X$};
				\node (b) at (1,0) {$i$};
				\node (label1) at (0.5,-0.15) {\tiny$+n$};
				\node (label2) at (0.5,0.15) {\tiny$\leq$};
				\draw (a) to (b);
			\end{tikzpicture}
		\end{tabular}
	\end{center}
\end{frame}
% ---
\begin{frame}
	\frametitle{Formen von Kanten (Forts.)}
	\begin{center}
		\begin{tabular}{lll}
			Fall & Constraint & Kante \\
			\hline
			\ref{case05} & $i + n \leq X$ &
			\begin{tikzpicture}[->,baseline=-0.25em]
				\node (a) at (0,0) {$i$};
				\node (b) at (1,0) {$X$};
				\node (label1) at (0.5,-0.15) {\tiny$-n$};
				\node (label2) at (0.5,0.15) {\tiny$\leq$};
				\draw (a) to (b);
			\end{tikzpicture}
			\\
			\ref{case06} & $i \leq X + n$ &
			\begin{tikzpicture}[->,baseline=-0.25em]
				\node (a) at (0,0) {$i$};
				\node (b) at (1,0) {$X$};
				\node (label1) at (0.5,-0.15) {\tiny$+n$};
				\node (label2) at (0.5,0.15) {\tiny$\leq$};
				\draw (a) to (b);
			\end{tikzpicture}
			\\
			\ref{case07} & $Y + n \leq X$ &
			\begin{tikzpicture}[->,baseline=-0.25em]
				\node (a) at (0,0) {$X$};
				\node (b) at (1,0) {$Y$};
				\node (label1) at (0.5,-0.15) {\tiny$-n$};
				\node (label2) at (0.5,0.15) {\tiny$\leq$};
				\draw (a) to (b);
			\end{tikzpicture}
			\\
			\ref{case08} & $Y \leq X + n$ &
			\begin{tikzpicture}[->,baseline=-0.25em]
				\node (a) at (0,0) {$X$};
				\node (b) at (1,0) {$Y$};
				\node (label1) at (0.5,-0.15) {\tiny$+n$};
				\node (label2) at (0.5,0.15) {\tiny$\leq$};
				\draw (a) to (b);
			\end{tikzpicture}
			\\
			\ref{case09} & $Y + n < X$    &
			\begin{tikzpicture}[->,baseline=-0.25em]
				\node (a) at (0,0) {$X$};
				\node (b) at (1,0) {$Y$};
				\node (label1) at (0.5,-0.15) {\tiny$-n$};
				\node (label2) at (0.5,0.15) {\tiny$<$};
				\draw (a) to (b);
			\end{tikzpicture}
			\\
			\ref{case10} & $Y < X + n$    &
			\begin{tikzpicture}[->,baseline=-0.25em]
				\node (a) at (0,0) {$X$};
				\node (b) at (1,0) {$Y$};
				\node (label1) at (0.5,-0.15) {\tiny$+n$};
				\node (label2) at (0.5,0.15) {\tiny$<$};
				\draw (a) to (b);
			\end{tikzpicture}
		\end{tabular}
	\end{center}
\end{frame}
% ---
\subsection{Komposition und Transitive Hülle}
\label{trans}
\begin{frame}
	\begin{itemize}
		\item Transitive Hülle enthält Informationen, die aus den Constraints gefolgert werden können.
		\item Floyd-Warshall benötigt Definition von $\min$ und $+$ von Kanten
		\item Komposition:
			\begin{center}
				\begin{tikzpicture}[->,baseline=-0.25em]
					\node (a) at (0,0) {$a$};
					\node (b) at (1,1) {$b$};
					\node (c) at (2,0) {$c$};
					\node (label1) at (1,-0.15) {\tiny$\alpha+\beta$};
					\node (label2) at (1,0.15) {\tiny$\leq$};
					\node (label3) at (0.6,0.4) {\tiny$\alpha$};
					\node (label4) at (0.38,0.62) {\tiny$\leq$};
					\node (label5) at (1.38,0.4) {\tiny$\beta$};
					\node (label6) at (1.6,0.62) {\tiny$\leq$};
					\draw (a) to (b);
					\draw (b) to (c);
					\draw[dashed] (a) to (c);
				\end{tikzpicture}
			\end{center}
		\item Minimum:
			\begin{center}
				\begin{tikzpicture}[scale=2,->,baseline=-0.25em]
					\node (a) at (0,0) {$a$}; => mininum (meet)
					\node (b) at (1,0) {$b$};
					\node (label1) at (0.5,0.1) {\tiny$\alpha$};
					\node (label2) at (0.5,0.25) {\tiny$\leq$};
					\node (label3) at (0.5,-0.25) {\tiny$\beta$};
					\node (label4) at (0.5,-0.1) {\tiny$\leq$};
					\draw[bend left] (a) to (b);
					\draw[bend right] (a) to (b);
				\end{tikzpicture}
				wird zu
				\begin{tikzpicture}[scale=2,->,baseline=-0.25em]
					\node (a) at (0,0) {$a$};
					\node (b) at (1,0) {$b$};
					\node (label1) at (0.5,-0.12) {\tiny$\min(\alpha,\beta)$};
					\node (label2) at (0.5,0.12) {\tiny$\leq$};
					\draw (a) to (b);
				\end{tikzpicture}
			\end{center}
		\item Offset beachten
		\item Reflexive Hülle bilden
	\end{itemize}
\end{frame}
% ---
\subsection{Widersprüche}
\label{cont}
\begin{frame}
	\begin{itemize}
		\item Vergleich mit Hypothesengraph
			\begin{itemize}
				\item Problem, wenn Constraint stärker als entsprechende Hypothese
			\end{itemize}
		\item Negative Zyklen im Constraintgraph
	\end{itemize}
\end{frame}
% ---
\subsection{Algebraische Struktur}
\label{alg}
\begin{frame}
	\begin{itemize}
		\item
			\begin{block}{$(L,\wedge)$ Beschränkter Halbverband}
				\begin{itemize}
					\item Assoziativität
					\item Kommutativität
					\item Idempotenz
					\item Neutrales Element
				\end{itemize}
			\end{block}
		\item
			\begin{block}{$(L,\wedge,\compose)$ Dioid (Halbring mit idempotentem $\meet$)}
				\begin{itemize}
					\item $(L,\wedge)$ kommutativer Monoid mit neutralem Element
					\item $(L,\compose)$ Monoid mit neutralem Element
					\item Distributivität von $\meet$ und $\compose$
					\item Komponieren ($\compose$) mit $\top$ ergibt $\top$
				\end{itemize}
			\end{block}
	\end{itemize}
\end{frame}
% ---
\section{Finden der Lösung}
\label{sol}
% ---
\subsection{Kleinste obere Schranke und größte untere Schranke}
\begin{frame}
	\begin{itemize}
		\item Kleinste obere Schranke ist festgelegt durch \emph{schärfste} ausgehende Kante
		\item Größte untere Schranke ist festgelegt durch \emph{schärfste} eingehende Kante
		\item Eingehende (bzw.\ ausgehende) Kanten anordnen, wenn möglich
	\end{itemize}
\end{frame}
% ---
\subsection{Zusammenhangskomponenten}
\begin{frame}
	\begin{itemize}
		\item Zusammenhangskomponenten getrennt betrachten beim Finden der Lösung
		\item Beim Bilden der Zusammenhangskomponenten $0$, $\infty$ und rigide Variablen weglassen
	\end{itemize}
\end{frame}
% ---
\begin{frame}
	\frametitle{Lösen}
	\begin{itemize}
		\item Wenn größte Lösung für $X$ gesucht, dann $X$ auf kleinste obere Schranke setzen
		\item Wenn kleinste Lösung für $X$ gesucht, dann $X$ auf größte untere Schranke setzen
	\end{itemize}
\end{frame}
% ---
\section{Erweitertes Problem}
\label{ext}
% ---
\subsection{Maxima}
\begin{frame}
	\begin{itemize}
		\item Zuerst vereinfachen
		\item Constraints der Form
			\[ \max(m,i_1+o_1,\dotsc,i_s+o_s) \leq i + n \]
			in einzelne Ungleichungen aufteilen
		\item Problem bei Constraints der Form
			\[ i + n \leq \max(m,i_1+o_1,\dotsc,i_s+o_s) \]
			da $i+n \leq$ als mindestens eines der der Argumente des Maximums sein muss % kann nicht einfach in Graphen abgebildet werden
		\item Solche Ungleichungen erstmal ignorieren und später auf Konsistenz überprüfen % existiert mindestens eine entsprechende Kante, die schärfer ist?
	\end{itemize}
\end{frame}
% ---
\subsection{Variablensignaturen}
\begin{frame}
	\begin{itemize}
		\item Zuerst: Alle flexiblen Variablen hängen von der gleichen Signatur von rigiden Variablen ab
		\item Jetzt erlauben wir auch $X\ i\ j\ k \leq Y\ i\ k\ j$
		\item Kanten zwischen solchen Knoten brauchen neues Merkmal
		\item Permutationsfunktion $\pi_{(\nu_1,\dotsc,\nu_n)}$, z.B.
			\[ \pi_{(1,3,2)}(i,j,k) = (i,k,j) \]
		\item Komposition kein Problem, aber
		\item Vergleich im Normalfall nicht möglich: Welche Kante ist schärfer?
			\begin{center}
				\begin{tikzpicture}[->,baseline=-0.25em]
					\node (x) at (0,0) {$X$};
					\node (y) at (3,0) {$Y$};
					\node (label1) at (1.5,-0.68) {\tiny$0$ $\leq$};
					\node (label2) at (1.5,0.68) {\tiny$0$ $\leq$};
					\node (label3) at (1.5,-0.35) {\tiny$\pi_{(1,2,3)}$};
					\node (label4) at (1.5,0.35) {\tiny$\pi_{(1,3,2)}$};
					\draw[bend right] (x) to (y);
					\draw[bend left] (x) to (y);
				\end{tikzpicture}
			\end{center}
	\end{itemize}
\end{frame}
% ---
\begin{frame}
	\frametitle{Beispiel}
	Constraints:
	\begin{align*}
		X i j &\leq Y j i \\
		X i j &\leq Y i j \\
		i &\leq X i j \\
		i &\leq Y i j
	\end{align*}
	und $\mathsf{pol}(X) = \mathsf{pol}(Y) = -$

	Hypothese:
	\[ i \leq j \]
\end{frame}
% ---
\begin{frame}
	Die transitive Hülle schaut folgendermaßen aus:
	\begin{center}
		\begin{tikzpicture}[->,scale=1.5,inner sep=0.15em]
			\node (0)  at (-1,-1) {$0$};
			\node (x) at (1,1) {$X$};
			\node (y) at (2,-1) {$Y$};
			\node (i) at (4,1) {$i$};
			\node (j) at (5,-1) {$j$};
			\node (oo) at (6,0) {$\infty$};
			\path
			(0) edge node {} (x)
			(0) edge node {} (y)
			(x) edge[bend left=20] node[xshift=-0.15cm, yshift=0.1cm, above right] {$(\leq,0,\pi_{(2,1)})$} (y)
			(x) edge[bend right=20] node[xshift=0.1cm, yshift=-0.35cm, left] {$(\leq,0,\pi_{(1,2)})$} (y)
			(i) edge node {} (x)
			(i) edge node {} (j)
			(i) edge node {} (y)
			(j) edge node {} (y)
			(x) edge[bend left=40] node {} (oo)
			(y) edge node {} (oo)
			;
		\end{tikzpicture}
	\end{center}
\end{frame}
% ---
\begin{frame}
	\begin{itemize}
		\item Größte untere Schranke für $Y$ ist $j$, kleinste untere Schranke ist $\infty$
		\item Lösung für $Y$: $j$
		\item Größte untere Schranke für $X$ ist $0$, kleinste untere Schranke ist $Y\ j\ i = i$ oder $Y\ i\ j = j$
		\item Lösung für $X$: $i$ und $j$
	\end{itemize}
\end{frame}
% ---
\section{Ende}
\begin{frame}
	\frametitle{Zusammenfassung}
	\begin{itemize}
		\item Lösen ist i.A. möglich, wenn keine Widersprüche auftreten
		\item Widersprüche werden erkannt
		\item Komplexität $\mathcal{O}(m n^3)$ bzw. $\mathcal{O}(m n^3 \log(n))$
	\end{itemize}
\end{frame}
% ---
\begin{frame}
	\begin{center}
		Vielen Dank für Ihre Aufmerksamkeit!
	\end{center}
\end{frame}
% ---
\end{document}
